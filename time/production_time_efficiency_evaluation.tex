\section{Production Time Efficiency Analysis}

Beyond power consumption optimization, the Cotton Candy Digital Twin demonstrated significant improvements in production time efficiency. This section examines the evolution of cotton candy manufacturing duration from initial learning phases to optimized operational states.

\subsection{Methodology}

Production time efficiency was measured using the \texttt{duration\_till\_handover} metric, which captures the complete manufacturing cycle from process initiation to final product delivery. This metric encompasses all critical manufacturing phases including heating, spinning, forming, and quality control processes. The analysis compared early learning iterations with mature operational performance, excluding statistical outliers exceeding 1.5 times the interquartile range.

\subsection{Results}

Table \ref{tab:production_efficiency} presents the comparative analysis of production time performance between initial learning and optimized operational phases.

\begin{table}[htbp]
\centering
\caption{Production Time Efficiency Optimization Results}
\label{tab:production_efficiency}
\begin{tabular}{lrrr}
\toprule
\textbf{Performance Metric} & \textbf{Initial Phase} & \textbf{Optimized Phase} & \textbf{Improvement} \\
& \textbf{(n=30)} & \textbf{(n=27)} & \\
\midrule
Average Production Time (s) & 231.81 ± 59.08 & 185.00 ± 20.48 & 20.2\% ↓ \\
Production Time (min) & 3.86 ± 0.98 & 3.08 ± 0.34 & 0.78 min ↓ \\
Time Variability (CV\%) & 25.5\% & 11.1\% & 56.5\% ↓ \\
Minimum Time (s) & 109.18 & 154.00 & -41.1\% \\
Maximum Time (s) & 345.43 & 218.00 & 36.9\% ↓ \\
Process Range (s) & 236.25 & 64.00 & 72.9\% ↓ \\
\bottomrule
\end{tabular}
\end{table}

\subsection{Statistical Validation}

The production time improvements demonstrated high statistical significance (p = 0.0003), confirming that the observed efficiency gains are not attributable to random variation. The coefficient of variation decreased from 25.5\% to 11.1\%, indicating a 56.5\% improvement in process consistency and predictability.

\subsection{Discussion}

The experimental results reveal three critical aspects of manufacturing optimization:

\textbf{Temporal Efficiency:} The system achieved a 20.2\% reduction in average production time, decreasing from 231.81 seconds (3.86 minutes) to 185.00 seconds (3.08 minutes) per cotton candy unit. This represents a time savings of 46.8 seconds per manufacturing cycle, which translates to significant throughput improvements in industrial applications.

\textbf{Process Predictability:} The dramatic reduction in time variability (56.5\% improvement in coefficient of variation) demonstrates enhanced process control and repeatability. This consistency is crucial for manufacturing quality assurance and production planning reliability.

\textbf{Operational Stability:} The reduction in process time range from 236.25 seconds to 64.00 seconds (72.9\% improvement) indicates that the system learned to eliminate extreme processing variations, leading to more stable and predictable manufacturing operations.

The concurrent achievement of both faster production times and reduced variability suggests that the digital twin successfully identified and eliminated inefficiencies while maintaining quality standards. The minimum production time increase from 109.18 to 154.00 seconds indicates that the system avoided overly aggressive optimization that might compromise product quality.

\subsection{Manufacturing Impact}

The production time optimization yields substantial operational benefits:

\begin{itemize}
\item \textbf{Throughput Enhancement:} 20.2\% reduction in cycle time enables proportional increases in production capacity
\item \textbf{Resource Utilization:} Improved time predictability facilitates better scheduling and resource allocation
\item \textbf{Quality Consistency:} Reduced process variability correlates with more uniform product characteristics
\item \textbf{Economic Benefits:} Time savings of 0.78 minutes per unit compounds to significant cost reductions in high-volume production
\end{itemize}

\subsection{Conclusions}

The Cotton Candy Digital Twin demonstrates exceptional learning capabilities in production time optimization, achieving statistically significant improvements in both efficiency (20.2\% faster) and consistency (56.5\% less variability). These results complement the previously demonstrated power consumption optimizations, establishing a comprehensive framework for sustainable and efficient manufacturing through digital twin technology. The simultaneous optimization of time, energy, and consistency metrics validates the holistic learning approach implemented in the digital twin system.