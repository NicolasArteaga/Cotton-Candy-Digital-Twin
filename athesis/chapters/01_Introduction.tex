\chapter{Introduction}
\label{sec:intro}

\section{Motivation}
\label{sec:intro:mo}

As industries evolve, the ability to optimize processes while minimizing waste has become increasingly important. Digital twins—virtual representations of physical systems—are transforming how processes can be monitored, analyzed, and improved. While reactive digital twins respond to events as they occur, providing immediate yet limited feedback, predictive digital twins forecast potential outcomes based on historical and real-time data, enabling proactive adjustments. A more recent frontier is the prescriptive digital twin, which goes beyond prediction by recommending concrete actions to achieve goals such as improving efficiency, reducing energy consumption, or enhancing product quality.

This thesis explores the development of a prescriptive digital twin for the Cotton Candy Automata, a robotic system we will design at the chair to automate cotton candy production. The process provides a controlled and measurable environment in which to evaluate the capabilities of a digital twin, with key parameters such as heating time, temperature, process duration, sugar amount, and energy usage offering tangible performance indicators.

The construction of the digital twin follows a bottom-up approach. Instead of starting from abstract models or historical approximations, we relied on physical measurements and sensor data from the automata. Through repeated experiments, we empirically identified parameters that govern the system's behavior and integrated them into the digital twin. This allowed us to directly model the causal impact of system configurations on outcomes, such as energy consumption, production time, and product quality.

A central objective of this thesis is to compare the prescriptive digital twin to the baseline robotic system without intelligence we created (the “unintelligent automata”). This comparison highlights the extent to which a data-driven, bottom-up digital twin can improve operational efficiency and decision-making. By analyzing metrics such as energy savings, time efficiency, and production quality, the research demonstrates the potential of prescriptive digital twins to contribute to more sustainable and effective industrial processes.

Ultimately, this work contributes to the broader understanding of how to design and implement prescriptive digital twins in practice. It shows how bottom-up methods—grounded in empirical data and real-time measurements—can be leveraged to transform physical processes into intelligent, self-optimizing systems.

\section{Research Questions}
\label{sec:intro:rq}
At least 3 questions. They should not be answerable yes/no. Questions should be
questions (1 sentence). But you are allowed to explain them in more detail. In
the explanation also tell how you plan to prove that your potential future
solution is good.

About 1 page.


-------------

There are four research questions that we want to answer throughout this thesis.

(1) \
\textbf{How does the implementation of a digital twin improve production quality, time efficiency, and energy savings in the Cotton Candy Automata?}
This question examines the extent to which the digital twin contributes to measurable performance gains. The analysis will compare key performance indicators of the digital twin against the baseline automata to quantify improvements in quality, time, and energy consumption.

(2) \textbf{What correlations and dependencies can be identified in the data collected from the Cotton Candy Automata, and how do they shape the process outcomes?}
By analyzing sensor data, this question aims to uncover the strongest relationships between variables such as heating time, spinning duration, and sugar input, and their effect on production quality. The findings will also be compared with the initial hypotheses formed during empirical testing to validate or refine our understanding of the system.

(3) \textbf{To what extent can the prescriptive digital twin provide actionable recommendations to optimize cotton candy production, including variables beyond current experimental control?}
This question assesses the digital twin’s capacity to recommend process adjustments. In particular, it considers environmental factors such as ambient temperature and humidity, which cannot be directly manipulated in the present setup but are likely to be influential in real-world production scenarios.


(4) \textbf{How transferable and generalizable are the methods and insights derived from the Cotton Candy Automata digital twin to other process environments?}
A digital twin is most valuable when its methods are not bound to a single technical context but can be applied across different domains. While the Cotton Candy Automata provides a controlled and measurable environment for testing, the broader relevance of this thesis depends on whether the bottom-up, sensor-driven methodology can extend to other processes.

\section{Contribution}
\label{sec:intro:con}

What will/have I do/done that nobody else has done before. About 1/2 page.

A digital twin from the intersection of a universal robots arm and a fucking cotton candy machine. How cool and random is this?
If somebody did it before I will jump from a cliff I tell you... xD

\section{Methodology}
\label{sec:intro:meth}



Design Science in Information System (Hevner). How are we doing research?

(1) Summary what design science is (it uses stakeholders, artefacts, steps,
...). 

(2) What are the stakeholders, artefacts, steps for MY case.
What does it mean for my thesis?

About 1.5 pages.

% Stichpunkte:
% - Use Hevner's framework to justify Design Science as methodology
% - You're building and evaluating a technical artefact (digital twin + quality assessment)
% - Real-world relevance: improving cotton candy production
% - Scientific rigor: structured measurement and evaluation
% - Matches Hevner's cycle: build → evaluate → improve

This work follows the Design Science Research (DSR) paradigm as defined by Hevner et al.~\cite{Hevner2004}. The DSR approach provides a structured framework for the development and evaluation of innovative artefacts that solve real-world problems in a rigorous and methodologically grounded manner.

In the context of this thesis, the central artefact is a digital twin system for a cotton candy production robot. This artefact comprises both the virtual representation of the physical process and the accompanying infrastructure for capturing environmental and process-related parameters (e.g., humidity, input sugar, and runtime), as well as metrics of product quality (e.g., volume, weight, and fluffiness).

The digital twin is iteratively refined through cycles of design and evaluation, in accordance with the DSR methodology. The system is deployed in a real-world production context and evaluated based on its ability to improve product quality and production efficiency under varying conditions. This dual focus on practical applicability and measurable improvement ensures that the artefact addresses both relevance and rigour — the core criteria of Hevner’s framework.

\section{Evaluation}
\label{sec:intro:ev}

How will I evaluate that my proposal is good. This ties into the research questions.

About 1 page.

\section{Structure}
\label{sec:intro:struct}

Which chapters will my thesis have, and what are they all about.
About 1/4 page.

So, we have Related Work where we go into existing literature on digital twins, food papers about cotton candy and data science book to build a solid foundation for our research. Important because we will use it a lot for our Data Collection, searching correlations and building models.
Next, we have our Solution Design, where we explain the stakeholders involved, and the artefacts we are creating. This section will detail how we are building and evaluating our digital twin. \todo{Rewrite this after you are finished}
Afterwards the Implementation Chapter, where we describe the technical details of our digital twin, including the data collection process, the modeling techniques used, and the integration with the physical cotton candy machine.
Then Evaluation, where we assess the effectiveness of our digital twin in optimizing the cotton candy production process, based on the research questions outlined earlier.
Afterwards in Discussion, we reflect on the implications of our findings, we answer the  the limitations of our approach, and potential avenues for future research.
And finally Conclusion, where we summarize the key contributions of our work, and its significance in the broader context of digital twin research and applications.
\todo{Rewrite this after you are finished}