\section{Power Consumption Optimization Analysis}

The Cotton Candy Digital Twin system demonstrates significant energy efficiency improvements through iterative learning and process optimization. This section presents a comparative analysis of power consumption patterns between early learning phases and optimized operational states.

\subsection{Methodology}

Power consumption data was collected from 157 manufacturing processes across 24 batches using smart plug sensors integrated into the production system. Energy consumption was calculated using trapezoidal numerical integration following the formula:

\begin{equation}
E = \sum_{i=1}^{n} \frac{P_{i-1} + P_i}{2} \cdot \Delta t_i
\end{equation}

where $E$ represents total energy consumption, $P_i$ is the power measurement at time $i$, and $\Delta t_i$ is the time interval between consecutive measurements.

\subsection{Experimental Results}

To evaluate the optimization performance, we compared the first 30 manufacturing iterations (learning phase) with the final 30 iterations (optimized phase), excluding statistical outliers that exceeded 1.5 times the interquartile range above the third quartile.

\begin{table}[htbp]
\centering
\caption{Power Consumption Optimization Results}
\label{tab:power_optimization}
\begin{tabular}{lrrr}
\toprule
\textbf{Performance Metric} & \textbf{Initial Phase} & \textbf{Optimized Phase} & \textbf{Improvement} \\
& \textbf{(n=30)} & \textbf{(n=27)} & \\
\midrule
Average Energy (Wh) & 78.41 ± 14.02 & 58.41 ± 5.16 & 25.5\% ↓ \\
Energy Variability (CV\%) & 17.9\% & 8.8\% & 50.8\% ↓ \\
Average Power (W) & 595.8 ± 56.3 & 561.1 ± 21.0 & 5.8\% ↓ \\
Power Variability (CV\%) & 9.4\% & 3.7\% & 60.6\% ↓ \\
Peak Power (W) & 974.4 ± 89.2 & 997.1 ± 42.8 & -2.3\% \\
Process Duration (s) & 468.8 ± 66.2 & 390.7 ± 27.9 & 16.7\% ↓ \\
Energy Efficiency (Wh/min) & 10.00 ± 1.26 & 8.96 ± 0.68 & 10.4\% ↑ \\
\bottomrule
\end{tabular}
\end{table}

\subsection{Statistical Validation}

Statistical significance testing using independent t-tests confirmed the reliability of observed improvements. Energy consumption reduction showed high statistical significance (p < 0.001), while power consumption optimization was also statistically significant (p = 0.004). The coefficient of variation decreased substantially for both energy (17.9\% to 8.8\%) and power (9.4\% to 3.7\%), indicating improved process consistency and predictability.

\subsection{Discussion}

The experimental results demonstrate three key optimization achievements:

\textbf{Energy Efficiency:} The system achieved a 25.5\% reduction in average energy consumption per process, decreasing from 78.41 Wh to 58.41 Wh. This improvement represents substantial operational cost savings and reduced environmental impact.

\textbf{Process Consistency:} Energy consumption variability decreased by 50.8\%, as measured by the coefficient of variation. This improvement indicates better process control and more predictable manufacturing outcomes, which directly correlates with product quality consistency.

\textbf{Operational Efficiency:} Process duration decreased by 16.7\% while simultaneously reducing energy consumption, representing a dual optimization that improves both throughput and efficiency. The energy efficiency metric (Wh/min) improved by 10.4\%, demonstrating faster processing with lower energy requirements.

The maintained peak power levels (997.1W vs. 974.4W) indicate that the optimization strategy focused on reducing sustained power consumption rather than limiting maximum power availability, preserving the system's capability to handle demanding manufacturing phases while minimizing overall energy usage.

\subsection{Conclusions}

The Cotton Candy Digital Twin successfully demonstrates learning-based power optimization in manufacturing processes. The statistically significant improvements in energy consumption (25.5\% reduction), process consistency (50.8\% variability reduction), and operational efficiency (16.7\% faster processing) validate the effectiveness of the digital twin approach for sustainable manufacturing optimization. These results establish a foundation for scalable energy-efficient manufacturing systems that can adapt and optimize autonomously through operational experience.