\chapter{Related Work}
\label{sec:rel}

Related Work - There is a girl with long dark hair doing a thesis on digital twin as a research analysis. ask her per discord

google scholar in-title digital twin / business processes, search queries that give you less than 20 papers and decide 1 or 2 papers in the end. How many papers did we eliminate? why? bc too

backgorund 1-2 sentences
-> Why am I doing something different to this related work

Final max 4-5 papers that really relate to what Im doing


THis one doesnt work bc, its for bridges and civil engineering
https://www.sciencedirect.com/science/article/pii/S0141029623017984

this one looks good :
https://www.sciencedirect.com/science/article/pii/S0360835224003620

\section{Thermal Study on Cotton Candy}
Cotton candy consists of spun sucrose that cools rapidly, forming a mostly amorphous structure — but:
	•	Over time, this amorphous state can convert into crystalline form.
	•	The ratio between crystalline and amorphous sucrose affects:
	•	Texture
	•	Stability
	•	Taste
	•	Shelf-life

% The paper investigates how the physical structure of cotton candy changes depending on how it is produced and stored, focusing especially on how much is amorphous vs crystalline sucrose.
% 	•	The author studies this using Differential Scanning Calorimetry (DSC), a thermal analysis technique to measure transitions like glass transitions, crystallization, and melting.
% 	•	The novelty: the study compares cotton candy spun in air vs in nitrogen atmosphere to see how different gas environments affect its physical and thermal properties.

% In essence, it’s a study of the physics of sugar in cotton candy — very relevant to your Cotton Candy Automata project, as this kind of data gives insight into the material side of your digital twin.

This paper provides very solid experimental data on how production parameters influence the physical structure (crystalline vs amorphous) of cotton candy.

Crystalline is \dots
Amorphous is \dots

% Because the amorphous vs crystalline ratio affects:
% 	•	Shelf life (amorphous recrystallizes over time)
% 	•	Stability (stickiness, collapse, hardening)
% 	•	Thermal behavior (different melting/glass transition points)
% 	•	Quality (texture, mouthfeel)

Knowing how production parameters (like gas environment) affect structure can inform optimal production recipes. We are not gonna compare CCA vs CCN, since we are always using Normal Air but
We learn about the importance of humidity, and want to use it for our Data Recollection since this is important For the Prescriptive twin design.

This helps with us taking the decision how to measure the quality of CC when doing data recollection and giving a note to the process.

What makes it difficult is that the changes are fine and little, and we dont really know if we are gonna be able to distinguish them, but we did research and will introducte this in the Solution Design.

% Smooth operatoooor


