\chapter{Discussion}
\label{sec:discussion}

\section{Measurement Limitations}
% Stichpunkte:
% - Volume measurements were approximate.
% - Shape irregularities and human error may affect accuracy.
% - Nevertheless, same method = reliable trends.
% - Future work could use more advanced shape capture.

While the estimation of product volume provided useful insights into structural quality, it is subject to several limitations. The irregular shape and delicate structure of cotton candy introduce measurement uncertainty, especially when relying on manual tools such as a ruler. Furthermore, the assumption of an ideal oblate spheroid shape simplifies the actual morphology, which may vary significantly across runs.

Despite these limitations, the same procedure was applied consistently throughout the experiments, ensuring the validity of comparative trends. For future work, more precise volume estimation techniques such as 3D scanning or photogrammetric analysis could be explored to capture the complex geometry of the product more accurately.
