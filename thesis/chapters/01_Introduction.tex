\chapter{Introduction}
\label{sec:intro}

For exposes, create this chapter, plus start with chapter 2 (Related Work).

\section{Motivation}
\label{sec:intro:mo}

Why are we doing it? About 1 page.

As industries evolve, the ability to optimize processes while minimizing waste has become increasingly important. Digital twins (virtual models of physical systems) are transforming how we monitor, analyze, and improve these processes. While reactive digital twins respond to events as they occur, providing immediate but limited feedback, predictive digital twins forecast potential outcomes based on historical and real-time data, allowing proactive adjustments. However, the growing focus on prescriptive digital twins introduces a new frontier: systems that not only predict outcomes but also recommend specific actions to achieve goals such as improving efficiency, reducing energy consumption, or enhancing product quality.

This thesis explores the development of prescriptive digital twins using the Cotton Candy Automata, a robotic system developed in the chair to automate cotton candy production, as a practical and measurable test case. This scenario provides an ideal environment for evaluating two distinct approaches to digital twin design—bottom-up and top-down—by analyzing key variables such as heating time, spinning duration, sugar amount, and energy usage. Each approach takes a unique perspective:

Bottom-Up Approach: Focuses on physical measurements and process models, leveraging real-time sensor data to guide system optimization.

Top-Down Approach: Relies on historical data and advanced computational methods, including interpolation, selection of closest historical values, and machine learning models like Recurrent Neural Networks (RNNs), to model and optimize the system.

The goal of this thesis is to provide a comprehensive evaluation of these two approaches, comparing their strengths, limitations, and suitability for different scenarios. By analyzing metrics such as energy savings, time efficiency, and production quality, the research aims to determine which method offers the greatest value (for this process/for specific process types?). Applying both methodologies to the same scenario enables a deeper understanding of their trade-offs and practical implications, offering insights that can guide future efforts in digital twin development.

Ultimately, this work contributes to the broader understanding of how to design and implement prescriptive digital twins, providing actionable recommendations for selecting the best approach based on system goals, constraints, and operational contexts.

\section{Research Questions}
\label{sec:intro:rq}

At least 3 questions. They should not be answerable yes/no. Questions should be
questions (1 sentence). But you are allowed to explain them in more detail. In
the explanation also tell how you plan to prove that your potential future
solution is good.

About 1 page.

Fexamples: 
(1) How can we design and implement prescriptive digital twins for the Cotton Candy Automata?
(2) MMM What are the strengths and limitations of each approach, and how do they impact key metrics such as energy savings, time efficiency, and production quality?

\section{Contribution}
\label{sec:intro:con}

What will/have I do/done that nobody else has done before. About 1/2 page.

\section{Methodology}
\label{sec:intro:meth}



Design Science in Information System (Hevner). How are we doing research?

(1) Summary what design science is (it uses stakeholders, artefacts, steps,
...). 

(2) What are the stakeholders, artefacts, steps for MY case.
What does it mean for my thesis?

About 1.5 pages.

% Stichpunkte:
% - Use Hevner's framework to justify Design Science as methodology
% - You're building and evaluating a technical artefact (digital twin + quality assessment)
% - Real-world relevance: improving cotton candy production
% - Scientific rigor: structured measurement and evaluation
% - Matches Hevner's cycle: build → evaluate → improve

This work follows the Design Science Research (DSR) paradigm as defined by Hevner et al.~\cite{Hevner2004}. The DSR approach provides a structured framework for the development and evaluation of innovative artefacts that solve real-world problems in a rigorous and methodologically grounded manner.

In the context of this thesis, the central artefact is a digital twin system for a cotton candy production robot. This artefact comprises both the virtual representation of the physical process and the accompanying infrastructure for capturing environmental and process-related parameters (e.g., humidity, input sugar, and runtime), as well as metrics of product quality (e.g., volume, weight, and fluffiness).

The digital twin is iteratively refined through cycles of design and evaluation, in accordance with the DSR methodology. The system is deployed in a real-world production context and evaluated based on its ability to improve product quality and production efficiency under varying conditions. This dual focus on practical applicability and measurable improvement ensures that the artefact addresses both relevance and rigour — the core criteria of Hevner’s framework.

\section{Evaluation}
\label{sec:intro:ev}

How will I evaluate that my proposal is good. This ties into the research questions.

About 1 page.

\section{Structure}
\label{sec:intro:struct}

Which chapters will my thesis have, and what are they all about.
About 1/4 page.
